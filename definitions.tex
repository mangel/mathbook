\documentclass{article}

\usepackage{amsthm}
\usepackage{amssymb}
\usepackage{amsfonts}
\usepackage{amsmath}
\usepackage[margin=0.5in]{geometry}

\newtheorem{theorem}{Theorem}[section]
\newtheorem{corollary}{Corollary}[theorem]
\newtheorem{lemma}[theorem]{Lemma}
\newtheorem*{remark}{Remark}

\begin{document}
\title{Mathematical definitions for everyone}
\author{Miguel Angel Gomez Barrera}

\maketitle

\section{Set}

\paragraph{Set} In mathematics, a set is a collection of well defined distinct objects, considered as an object in its own right.

\section{Function}
(MSC: 03E99)
\paragraph{Function} . Is a relation between sets in which one element of the set is related to another element of the second set.

$$f: A \mapsto B$$

Functions are composed by three things:

\begin{itemize}
    \item The domain.
    \item The codomain or range.
    \item The correspondence rule.
\end{itemize}

$$f(n):=$$

\paragraph{Domain}
\paragraph{Codomain or Range}

\begin{theorem}[Taylor Polinomial] Given some function $f$ with derivatives of order $n$ on the point $x=0$. Exists a only one polynomial $P$ on degree less or equal of $n$ which satisfies the $n + 1$ conditions:
	$$P(0) = f(0), P'(0)=f'(0), \dots, P^n = f^{(n)}(0)$$
, and given by the formula:
	$$P(x) = \sum_{k=0}^{n} \frac{f^{(k)}(0)}{x!}x^{k}$$
where $x \neq x_0$.
\end{theorem}

\end{document}