\documentclass{article}

\usepackage{amsmath}
\usepackage{amsfonts}
\usepackage{amssymb}
\usepackage{amsthm}
\usepackage{graphicx}
\usepackage[margin=0.5in]{geometry}
\usepackage[utf8]{inputenc}

\newtheorem{theorem}{Theorem}[section]
\newtheorem{corollary}{Corollary}[theorem]
\newtheorem{lemma}[theorem]{Lemma}
\newtheorem{definition}{Definition}[section]

\begin{document}
\paragraph{}Things to study:
\begin{itemize}
    \item Point.
    \item Distance between points.
    \item Line.
    \item Plane.
    \item 
\end{itemize}
\section{Linear algebra}
\begin{definition}{(Dot).}
is an abstract idea to represent a place on a nth dimension.
\end{definition}
\paragraph{} On one dimension we can say that a dot is an $x \in S$ on which $S$ is a set. Similarly on the two and three dimension we have:
$$(x,y) \in S^{2},$$
$$(x,y,z) \in S^{3},$$
In general we will be working from now on with the real numbers: $\mathbb{R}^n$, where $n$ is the dimension.
\subsection{Operations}
\begin{definition}{Addition (+)}
Given two points $P_1$ and $P_2$, we say that the addition of these two points will be equal to some other point. And it is represented as:
$$P_1 + P_2 = P_3,$$
also a more general on nth dimension sum will be:
$$(x_1, y_1, z_1, \dots, m_1) + (x_2, y_2, z_2, \dots, m_2) = (x_1 + x_2, y_1 + y_2, z_1 + z_2, \dots, m_1 + m_2)$$
\end{definition}
\begin{definition}{Dot Product ($\cdot$)}
Given two points $P_1$ and $P_2$, we say that the addition of these two points will be equal to some scalar k. Then
$$P_1 = (x_1, y_1, z_1, \dots, m_1), P_2 = (x_2, y_2, z_2, \dots, m_2)$$
\begin{align*}
    P_1 \cdot P_2 &= P_3\\
    (x_1, y_1, z_1, \dots, m_1) \cdot (x_2, y_2, z_2, \dots, m_2) &=  x_1 x_2 + y_1 y_2 + z_1 z_2 + \dots + m_1 m_2
\end{align*}
\end{definition}
\begin{definition}{Scalar product}
The scalar product is...
\end{definition}
\end{document}
